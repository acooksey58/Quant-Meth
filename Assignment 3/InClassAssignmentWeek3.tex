% Options for packages loaded elsewhere
\PassOptionsToPackage{unicode}{hyperref}
\PassOptionsToPackage{hyphens}{url}
%
\documentclass[
]{article}
\usepackage{lmodern}
\usepackage{amssymb,amsmath}
\usepackage{ifxetex,ifluatex}
\ifnum 0\ifxetex 1\fi\ifluatex 1\fi=0 % if pdftex
  \usepackage[T1]{fontenc}
  \usepackage[utf8]{inputenc}
  \usepackage{textcomp} % provide euro and other symbols
\else % if luatex or xetex
  \usepackage{unicode-math}
  \defaultfontfeatures{Scale=MatchLowercase}
  \defaultfontfeatures[\rmfamily]{Ligatures=TeX,Scale=1}
\fi
% Use upquote if available, for straight quotes in verbatim environments
\IfFileExists{upquote.sty}{\usepackage{upquote}}{}
\IfFileExists{microtype.sty}{% use microtype if available
  \usepackage[]{microtype}
  \UseMicrotypeSet[protrusion]{basicmath} % disable protrusion for tt fonts
}{}
\makeatletter
\@ifundefined{KOMAClassName}{% if non-KOMA class
  \IfFileExists{parskip.sty}{%
    \usepackage{parskip}
  }{% else
    \setlength{\parindent}{0pt}
    \setlength{\parskip}{6pt plus 2pt minus 1pt}}
}{% if KOMA class
  \KOMAoptions{parskip=half}}
\makeatother
\usepackage{xcolor}
\IfFileExists{xurl.sty}{\usepackage{xurl}}{} % add URL line breaks if available
\IfFileExists{bookmark.sty}{\usepackage{bookmark}}{\usepackage{hyperref}}
\hypersetup{
  pdftitle={in class\_assign3},
  hidelinks,
  pdfcreator={LaTeX via pandoc}}
\urlstyle{same} % disable monospaced font for URLs
\usepackage[margin=1in]{geometry}
\usepackage{color}
\usepackage{fancyvrb}
\newcommand{\VerbBar}{|}
\newcommand{\VERB}{\Verb[commandchars=\\\{\}]}
\DefineVerbatimEnvironment{Highlighting}{Verbatim}{commandchars=\\\{\}}
% Add ',fontsize=\small' for more characters per line
\usepackage{framed}
\definecolor{shadecolor}{RGB}{248,248,248}
\newenvironment{Shaded}{\begin{snugshade}}{\end{snugshade}}
\newcommand{\AlertTok}[1]{\textcolor[rgb]{0.94,0.16,0.16}{#1}}
\newcommand{\AnnotationTok}[1]{\textcolor[rgb]{0.56,0.35,0.01}{\textbf{\textit{#1}}}}
\newcommand{\AttributeTok}[1]{\textcolor[rgb]{0.77,0.63,0.00}{#1}}
\newcommand{\BaseNTok}[1]{\textcolor[rgb]{0.00,0.00,0.81}{#1}}
\newcommand{\BuiltInTok}[1]{#1}
\newcommand{\CharTok}[1]{\textcolor[rgb]{0.31,0.60,0.02}{#1}}
\newcommand{\CommentTok}[1]{\textcolor[rgb]{0.56,0.35,0.01}{\textit{#1}}}
\newcommand{\CommentVarTok}[1]{\textcolor[rgb]{0.56,0.35,0.01}{\textbf{\textit{#1}}}}
\newcommand{\ConstantTok}[1]{\textcolor[rgb]{0.00,0.00,0.00}{#1}}
\newcommand{\ControlFlowTok}[1]{\textcolor[rgb]{0.13,0.29,0.53}{\textbf{#1}}}
\newcommand{\DataTypeTok}[1]{\textcolor[rgb]{0.13,0.29,0.53}{#1}}
\newcommand{\DecValTok}[1]{\textcolor[rgb]{0.00,0.00,0.81}{#1}}
\newcommand{\DocumentationTok}[1]{\textcolor[rgb]{0.56,0.35,0.01}{\textbf{\textit{#1}}}}
\newcommand{\ErrorTok}[1]{\textcolor[rgb]{0.64,0.00,0.00}{\textbf{#1}}}
\newcommand{\ExtensionTok}[1]{#1}
\newcommand{\FloatTok}[1]{\textcolor[rgb]{0.00,0.00,0.81}{#1}}
\newcommand{\FunctionTok}[1]{\textcolor[rgb]{0.00,0.00,0.00}{#1}}
\newcommand{\ImportTok}[1]{#1}
\newcommand{\InformationTok}[1]{\textcolor[rgb]{0.56,0.35,0.01}{\textbf{\textit{#1}}}}
\newcommand{\KeywordTok}[1]{\textcolor[rgb]{0.13,0.29,0.53}{\textbf{#1}}}
\newcommand{\NormalTok}[1]{#1}
\newcommand{\OperatorTok}[1]{\textcolor[rgb]{0.81,0.36,0.00}{\textbf{#1}}}
\newcommand{\OtherTok}[1]{\textcolor[rgb]{0.56,0.35,0.01}{#1}}
\newcommand{\PreprocessorTok}[1]{\textcolor[rgb]{0.56,0.35,0.01}{\textit{#1}}}
\newcommand{\RegionMarkerTok}[1]{#1}
\newcommand{\SpecialCharTok}[1]{\textcolor[rgb]{0.00,0.00,0.00}{#1}}
\newcommand{\SpecialStringTok}[1]{\textcolor[rgb]{0.31,0.60,0.02}{#1}}
\newcommand{\StringTok}[1]{\textcolor[rgb]{0.31,0.60,0.02}{#1}}
\newcommand{\VariableTok}[1]{\textcolor[rgb]{0.00,0.00,0.00}{#1}}
\newcommand{\VerbatimStringTok}[1]{\textcolor[rgb]{0.31,0.60,0.02}{#1}}
\newcommand{\WarningTok}[1]{\textcolor[rgb]{0.56,0.35,0.01}{\textbf{\textit{#1}}}}
\usepackage{graphicx,grffile}
\makeatletter
\def\maxwidth{\ifdim\Gin@nat@width>\linewidth\linewidth\else\Gin@nat@width\fi}
\def\maxheight{\ifdim\Gin@nat@height>\textheight\textheight\else\Gin@nat@height\fi}
\makeatother
% Scale images if necessary, so that they will not overflow the page
% margins by default, and it is still possible to overwrite the defaults
% using explicit options in \includegraphics[width, height, ...]{}
\setkeys{Gin}{width=\maxwidth,height=\maxheight,keepaspectratio}
% Set default figure placement to htbp
\makeatletter
\def\fps@figure{htbp}
\makeatother
\setlength{\emergencystretch}{3em} % prevent overfull lines
\providecommand{\tightlist}{%
  \setlength{\itemsep}{0pt}\setlength{\parskip}{0pt}}
\setcounter{secnumdepth}{-\maxdimen} % remove section numbering

\title{in class\_assign3}
\author{}
\date{\vspace{-2.5em}}

\begin{document}
\maketitle

SUCCESS OF LEADER ASSASSINATION AS A NATURAL EXPERIMENT

One longstanding debate in the study of international relations concerns
the question of whether individual political leaders can make a
difference. Some emphasize that leaders with different ideologies and
personalities can significantly affect the course of a nation. Others
argue that political leaders are severely constrained by historical and
institutional forces. Did individuals like Hitler, Mao, Roosevelt, and
Churchill make a big difference? The difficulty of empirically testing
these arguments stems from the fact that the change of leadership is not
random and there are many confounding factors to be adjusted for. In
this exercise, we consider a natural experiment in which the success or
failure of assassination attempts is assumed to be essentially random.7
Each observation of the CSV data set leaders.csv contains information
about an assassination attempt. Table 2.8 presents the names and
descriptions of variables in this leader assassination data set. The
polity variable represents the so-called polity score from the Polity
Project. The Polity Project systematically documents and quantifies the
regime types of all countries in the world from 1800. The polity score
is a 21-point scale ranging from −10 (hereditary monarchy) to 10
(consolidated democracy). The result variable is a 10-category factor
variable describing the result of each assassination attempt.

\begin{enumerate}
\def\labelenumi{\arabic{enumi}.}
\tightlist
\item
  How many assassination attempts are recorded in the data? How many
  countries experience at least one leader assassination attempt? (The
  unique() function, What is the average number of such attempts (per
  year) among these countries?
\end{enumerate}

\begin{Shaded}
\begin{Highlighting}[]
\NormalTok{leaders <-}\StringTok{ }\KeywordTok{read.csv}\NormalTok{(}\StringTok{"leaders.csv"}\NormalTok{)}
\KeywordTok{dim}\NormalTok{(leaders)}
\end{Highlighting}
\end{Shaded}

\begin{verbatim}
## [1] 250  11
\end{verbatim}

\begin{Shaded}
\begin{Highlighting}[]
\KeywordTok{summary}\NormalTok{(leaders}\OperatorTok{$}\NormalTok{result)}
\end{Highlighting}
\end{Shaded}

\begin{verbatim}
##    Length     Class      Mode 
##       250 character character
\end{verbatim}

\begin{Shaded}
\begin{Highlighting}[]
\NormalTok{(}\StringTok{"there were 250 assissination attempts"}\NormalTok{)}
\end{Highlighting}
\end{Shaded}

\begin{verbatim}
## [1] "there were 250 assissination attempts"
\end{verbatim}

\begin{Shaded}
\begin{Highlighting}[]
\KeywordTok{unique}\NormalTok{(leaders}\OperatorTok{$}\NormalTok{country)}
\end{Highlighting}
\end{Shaded}

\begin{verbatim}
##  [1] "Afghanistan"       "Albania"           "Algeria"          
##  [4] "Argentina"         "Australia"         "Austria"          
##  [7] "Belgium"           "Bhutan"            "Bolivia"          
## [10] "Brazil"            "Burundi"           "Bulgaria"         
## [13] "Cambodia"          "Canada"            "Ivory Coast"      
## [16] "Chad"              "Chile"             "China"            
## [19] "Colombia"          "Congo Brazzaville" "Costa Rica"       
## [22] "Cuba"              "Cyprus"            "Czechoslovakia"   
## [25] "Dominican Rep"     "Congo Kinshasa"    "Ecuador"          
## [28] "Egypt"             "Ethiopia"          "France"           
## [31] "Ghana"             "Germany"           "Greece"           
## [34] "Georgia"           "Guatemala"         "Guinea"           
## [37] "Haiti"             "Honduras"          "India"            
## [40] "Indonesia"         "Iran"              "Iraq"             
## [43] "Israel"            "Italy"             "Jordan"           
## [46] "Japan"             "Kenya"             "Kuwait"           
## [49] "Liberia"           "Lebanon"           "Libya"            
## [52] "Madagascar"        "Mexico"            "Myanmar (Burma)"  
## [55] "Nepal"             "Nicaragua"         "Niger"            
## [58] "Netherlands"       "Oman"              "Pakistan"         
## [61] "Panama"            "Peru"              "Poland"           
## [64] "Portugal"          "Korea South"       "Russia"           
## [67] "Vietnam South"     "Rwanda"            "South Africa"     
## [70] "El Salvador"       "Saudi Arabia"      "Senegal"          
## [73] "Somalia"           "Spain"             "Sri Lanka"        
## [76] "Sudan"             "Sweden"            "Syria"            
## [79] "Togo"              "Turkey"            "Uganda"           
## [82] "United Kingdom"    "Uruguay"           "United States"    
## [85] "Uzbekistan"        "Venezuela"         "Yemen North"      
## [88] "Yugoslavia"
\end{verbatim}

\begin{Shaded}
\begin{Highlighting}[]
\NormalTok{(}\StringTok{"there are 88 countries who experienced assissination attempts"}\NormalTok{)}
\end{Highlighting}
\end{Shaded}

\begin{verbatim}
## [1] "there are 88 countries who experienced assissination attempts"
\end{verbatim}

\begin{Shaded}
\begin{Highlighting}[]
\KeywordTok{mean}\NormalTok{(}\KeywordTok{tapply}\NormalTok{(leaders}\OperatorTok{$}\NormalTok{country, leaders}\OperatorTok{$}\NormalTok{year, length))}
\end{Highlighting}
\end{Shaded}

\begin{verbatim}
## [1] 2.45098
\end{verbatim}

\begin{Shaded}
\begin{Highlighting}[]
\NormalTok{(}\StringTok{"there are 2.45 attempts per year "}\NormalTok{)}
\end{Highlighting}
\end{Shaded}

\begin{verbatim}
## [1] "there are 2.45 attempts per year "
\end{verbatim}

\begin{enumerate}
\def\labelenumi{\arabic{enumi}.}
\setcounter{enumi}{1}
\tightlist
\item
  Create a new binary variable named success that is equal to 1 if a
  leader dies from the attack and 0 if the leader survives. Store this
  new variable as part of the original data frame. What is the overall
  success rate of leader assassination? Does the result speak to the
  validity of the assumption that the success of assassination attempts
  is randomly determined?
\end{enumerate}

\begin{Shaded}
\begin{Highlighting}[]
\NormalTok{result <-}\StringTok{ }\KeywordTok{factor}\NormalTok{(leaders}\OperatorTok{$}\NormalTok{result)}
\KeywordTok{levels}\NormalTok{(result)}
\end{Highlighting}
\end{Shaded}

\begin{verbatim}
##  [1] "dies between a day and a week"              
##  [2] "dies between a week and a month"            
##  [3] "dies within a day after the attack"         
##  [4] "dies, timing unknown"                       
##  [5] "hospitalization but no permanent disability"
##  [6] "not wounded"                                
##  [7] "plot stopped"                               
##  [8] "survives but wounded severely"              
##  [9] "survives, whether wounded unknown"          
## [10] "wounded lightly"
\end{verbatim}

\begin{Shaded}
\begin{Highlighting}[]
\NormalTok{(}\StringTok{"levels 1-4 are dead. 5-10 are not dead."}\NormalTok{)}
\end{Highlighting}
\end{Shaded}

\begin{verbatim}
## [1] "levels 1-4 are dead. 5-10 are not dead."
\end{verbatim}

\begin{Shaded}
\begin{Highlighting}[]
\NormalTok{leaders}\OperatorTok{$}\NormalTok{success[leaders}\OperatorTok{$}\NormalTok{result }\OperatorTok{==}\StringTok{ "dies between a day and a week"} 
               \OperatorTok{|}\StringTok{ }\NormalTok{leaders}\OperatorTok{$}\NormalTok{result }\OperatorTok{==}\StringTok{ "dies between a week and a month"} 
               \OperatorTok{|}\StringTok{ }\NormalTok{leaders}\OperatorTok{$}\NormalTok{result }\OperatorTok{==}\StringTok{ "dies within a day after the attack"}
               \OperatorTok{|}\StringTok{ }\NormalTok{leaders}\OperatorTok{$}\NormalTok{result }\OperatorTok{==}\StringTok{ "dies, timing unknown"}
\NormalTok{               ] <-}\StringTok{ }\DecValTok{1}


\NormalTok{leaders}\OperatorTok{$}\NormalTok{success[leaders}\OperatorTok{$}\NormalTok{result }\OperatorTok{==}\StringTok{ "hospitalization but no permanent disability"} 
               \OperatorTok{|}\StringTok{ }\NormalTok{leaders}\OperatorTok{$}\NormalTok{result }\OperatorTok{==}\StringTok{ "not wounded"}
               \OperatorTok{|}\StringTok{ }\NormalTok{leaders}\OperatorTok{$}\NormalTok{result }\OperatorTok{==}\StringTok{ "plot stopped"}
               \OperatorTok{|}\StringTok{ }\NormalTok{leaders}\OperatorTok{$}\NormalTok{result }\OperatorTok{==}\StringTok{ "survives but wounded severely"}
               \OperatorTok{|}\StringTok{ }\NormalTok{leaders}\OperatorTok{$}\NormalTok{result }\OperatorTok{==}\StringTok{ "survives, whether wounded unknown"}
               \OperatorTok{|}\StringTok{ }\NormalTok{leaders}\OperatorTok{$}\NormalTok{result }\OperatorTok{==}\StringTok{ "wounded lightly"}
\NormalTok{               ] <-}\StringTok{ }\DecValTok{0}
\KeywordTok{mean}\NormalTok{(leaders}\OperatorTok{$}\NormalTok{success)}
\end{Highlighting}
\end{Shaded}

\begin{verbatim}
## [1] 0.216
\end{verbatim}

\begin{Shaded}
\begin{Highlighting}[]
\NormalTok{(}\StringTok{"there is a 21.6% success rate"}\NormalTok{)}
\end{Highlighting}
\end{Shaded}

\begin{verbatim}
## [1] "there is a 21.6% success rate"
\end{verbatim}

\begin{Shaded}
\begin{Highlighting}[]
\NormalTok{(}\StringTok{"the success rate seems to not be random. Meaning that it is not valid assumption"}\NormalTok{)}
\end{Highlighting}
\end{Shaded}

\begin{verbatim}
## [1] "the success rate seems to not be random. Meaning that it is not valid assumption"
\end{verbatim}

\begin{enumerate}
\def\labelenumi{\arabic{enumi}.}
\setcounter{enumi}{2}
\tightlist
\item
  Investigate whether the average polity score over three years prior to
  an assassination attempt differs on average between successful and
  failed attempts. Also, examine whether there is any difference in the
  age of targeted leaders between successful and failed attempts.
  Briefly interpret the results in light of the validity of the
  aforementioned assumption.
\end{enumerate}

\begin{Shaded}
\begin{Highlighting}[]
\KeywordTok{tapply}\NormalTok{(leaders}\OperatorTok{$}\NormalTok{politybefore, leaders}\OperatorTok{$}\NormalTok{success, mean)}
\end{Highlighting}
\end{Shaded}

\begin{verbatim}
##          0          1 
## -1.7431973 -0.7037037
\end{verbatim}

\begin{Shaded}
\begin{Highlighting}[]
\NormalTok{(}\StringTok{"The difference is 1.04 which does not appear trivial"}\NormalTok{)}
\end{Highlighting}
\end{Shaded}

\begin{verbatim}
## [1] "The difference is 1.04 which does not appear trivial"
\end{verbatim}

\begin{Shaded}
\begin{Highlighting}[]
\KeywordTok{tapply}\NormalTok{(leaders}\OperatorTok{$}\NormalTok{age, leaders}\OperatorTok{$}\NormalTok{success, mean)}
\end{Highlighting}
\end{Shaded}

\begin{verbatim}
##        0        1 
## 52.71429 56.46296
\end{verbatim}

\begin{Shaded}
\begin{Highlighting}[]
\NormalTok{(}\StringTok{"successful aimed leaders is 3 years higher"}\NormalTok{)}
\end{Highlighting}
\end{Shaded}

\begin{verbatim}
## [1] "successful aimed leaders is 3 years higher"
\end{verbatim}

\begin{enumerate}
\def\labelenumi{\arabic{enumi}.}
\setcounter{enumi}{3}
\tightlist
\item
  Repeat the same analysis as in the previous question, but this time
  using the country's experience of civil and international war. Create
  a new binary variable in the data frame called warbefore. Code the
  variable such that it is equal to 1 if a country is in either civil or
  international war during the three years prior to an assassination
  attempt. Provide a brief interpretation of the result.
\end{enumerate}

\begin{Shaded}
\begin{Highlighting}[]
\NormalTok{leaders}\OperatorTok{$}\NormalTok{warbefore[leaders}\OperatorTok{$}\NormalTok{civilwarbefore }\OperatorTok{==}\StringTok{ }\DecValTok{1}
                       \OperatorTok{|}\StringTok{ }\NormalTok{leaders}\OperatorTok{$}\NormalTok{interwarbefore }\OperatorTok{==}\StringTok{ }\DecValTok{1}\NormalTok{] <-}\StringTok{ }\DecValTok{1} 

\NormalTok{leaders}\OperatorTok{$}\NormalTok{warbefore[leaders}\OperatorTok{$}\NormalTok{civilwarbefore }\OperatorTok{!=}\StringTok{ }\DecValTok{1} \OperatorTok{&}\StringTok{ }\NormalTok{leaders}\OperatorTok{$}\NormalTok{interwarbefore }\OperatorTok{!=}\StringTok{ }\DecValTok{1}\NormalTok{] <-}\StringTok{ }\DecValTok{0} 

\KeywordTok{summary}\NormalTok{(leaders}\OperatorTok{$}\NormalTok{warbefore)}
\end{Highlighting}
\end{Shaded}

\begin{verbatim}
##    Min. 1st Qu.  Median    Mean 3rd Qu.    Max. 
##   0.000   0.000   0.000   0.368   1.000   1.000
\end{verbatim}

\begin{Shaded}
\begin{Highlighting}[]
\NormalTok{war3 <-}\StringTok{ }\KeywordTok{subset}\NormalTok{(leaders, leaders}\OperatorTok{$}\NormalTok{warbefore }\OperatorTok{==}\StringTok{ }\DecValTok{1}\NormalTok{)}

\KeywordTok{tapply}\NormalTok{(war3}\OperatorTok{$}\NormalTok{politybefore, war3}\OperatorTok{$}\NormalTok{success, mean)}
\end{Highlighting}
\end{Shaded}

\begin{verbatim}
##          0          1 
## -1.7534247 -0.1929825
\end{verbatim}

\begin{Shaded}
\begin{Highlighting}[]
\KeywordTok{tapply}\NormalTok{(war3}\OperatorTok{$}\NormalTok{age, war3}\OperatorTok{$}\NormalTok{success, mean)}
\end{Highlighting}
\end{Shaded}

\begin{verbatim}
##        0        1 
## 53.13699 57.26316
\end{verbatim}

\begin{Shaded}
\begin{Highlighting}[]
\NormalTok{(}\StringTok{"countries that experience war have a higher polity"}\NormalTok{)}
\end{Highlighting}
\end{Shaded}

\begin{verbatim}
## [1] "countries that experience war have a higher polity"
\end{verbatim}

\begin{Shaded}
\begin{Highlighting}[]
\NormalTok{(}\StringTok{"age of ploted leaders is higher among successes"}\NormalTok{)}
\end{Highlighting}
\end{Shaded}

\begin{verbatim}
## [1] "age of ploted leaders is higher among successes"
\end{verbatim}

\begin{enumerate}
\def\labelenumi{\arabic{enumi}.}
\setcounter{enumi}{4}
\tightlist
\item
  Does successful leader assassination cause democratization? Does
  successful leader assassination lead countries to war? When analyzing
  these data, be sure to state your assumptions and provide a brief
  interpretation of the results.
\end{enumerate}

\begin{Shaded}
\begin{Highlighting}[]
\NormalTok{leaders}\OperatorTok{$}\NormalTok{warafter[leaders}\OperatorTok{$}\NormalTok{civilwarafter }\OperatorTok{==}\StringTok{ }\DecValTok{1}
                 \OperatorTok{|}\StringTok{ }\NormalTok{leaders}\OperatorTok{$}\NormalTok{interwarafter }\OperatorTok{==}\StringTok{ }\DecValTok{1}\NormalTok{] <-}\StringTok{ }\DecValTok{1}

\NormalTok{leaders}\OperatorTok{$}\NormalTok{warafter[leaders}\OperatorTok{$}\NormalTok{civilwarafter }\OperatorTok{!=}\StringTok{ }\DecValTok{1} \OperatorTok{&}\StringTok{ }\NormalTok{leaders}\OperatorTok{$}\NormalTok{interwarafter }\OperatorTok{!=}\StringTok{ }\DecValTok{1}\NormalTok{] <-}\StringTok{ }\DecValTok{0} 

\NormalTok{war3b <-}\StringTok{ }\KeywordTok{subset}\NormalTok{(leaders, }\DataTypeTok{subset =}\NormalTok{ (warafter }\OperatorTok{==}\StringTok{ }\DecValTok{1}\NormalTok{))}

\KeywordTok{tapply}\NormalTok{(leaders}\OperatorTok{$}\NormalTok{warafter, leaders}\OperatorTok{$}\NormalTok{success, mean)}
\end{Highlighting}
\end{Shaded}

\begin{verbatim}
##         0         1 
## 0.2959184 0.2037037
\end{verbatim}

\begin{Shaded}
\begin{Highlighting}[]
\KeywordTok{tapply}\NormalTok{(leaders}\OperatorTok{$}\NormalTok{polityafter, leaders}\OperatorTok{$}\NormalTok{success, mean)}
\end{Highlighting}
\end{Shaded}

\begin{verbatim}
##          0          1 
## -1.8945578 -0.7623457
\end{verbatim}

\begin{Shaded}
\begin{Highlighting}[]
\NormalTok{(}\StringTok{"After assassin success war breaking out is 20%"}\NormalTok{)}
\end{Highlighting}
\end{Shaded}

\begin{verbatim}
## [1] "After assassin success war breaking out is 20%"
\end{verbatim}

\begin{Shaded}
\begin{Highlighting}[]
\NormalTok{(}\StringTok{"While its rate after failure is 30%"}\NormalTok{)}
\end{Highlighting}
\end{Shaded}

\begin{verbatim}
## [1] "While its rate after failure is 30%"
\end{verbatim}

\end{document}
